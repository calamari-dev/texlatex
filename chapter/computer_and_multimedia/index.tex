\documentclass[../../index]{subfiles}

\begin{document}
\chapter{コンピュータとマルチメディア}
\section{文字コード}
コンピュータにおいて,テキストを記録するには,テキストを数値の列に変換しなければならない.
この変換方式のことを\termdef{文字コード}{もじこーど}という.
文字コードは複数あり,それぞれ利用できる文字や変換方式などが異なる.

\begin{floatingfigure}{5cm}
  \centering
  \tcbincludegraphics[myframe,width=5cm]{notepad.png}
  \caption{Windows 10のメモ帳}
\end{floatingfigure}

たとえば,最新のWindows 10にあるメモ帳でテキストを保存すると,UTF-8という方式で
文字列が数値の列に変換される.この他にも,ソフトウェアに応じてASCII,Shift\_JISなど,
さまざまな文字コードが利用されている.

現在では\termdef{Unicode}{unicode}という文字コードの規格が主流になりつつある.
先述のUTF-8も,このUnicodeにおいて定義されている.
この節では文字コードの代表例として,Unicodeについて解説する.
なお,以下に示す用語は文字コードによっては異なる名前で呼ばれることもあるし,対応する概念が無いこともある.

文字コードが扱える文字の全体集合を\termdef{文字集合}{もじしゅうごう}という.
文字集合の各要素には固有の非負整数が割り当てられており,
その対応関係をUnicodeでは\termdef{符号化文字集合}{ふごうかもじしゅうごう}と呼ぶ
\footnote{対応関係なのに集合?と思うのももっともだが,公式資料には確かに「a mapping from an abstract character repertoire to a set of nonnegative integers」\cite{Whistler2008}とある.}.

符号化文字集合(の像)は,集合\(S=\Set{0_{(16)},\dots,\literal{10FFFF}_{(16)}}\)の真部分集合である.
\(S\)を\termdef{符号空間}{ふごうくうかん}といい,\(S\)の要素を\termdef{符号位置}{ふごういち}という.
符号位置は,十六進数の先頭に「U+」を付けて「U+304B」のようにして表される.

符号化文字集合は,数値を並べたときに区切れ目がただ1通りに定まるかどうかを考慮していない.
そこで,あとから区切れ目が分かるように,符号化文字集合(の像)に属する数値を整数の組\footnote{正確には1バイトの組である.}(\termdef{符号単位}{ふごうたんい})へと変換する規則が必要になる.
この規則を\termdef{文字符号化形式}{もじふごうかけいしき}といい,UTF-8,UTF-16,UTF-32の3つが定義されている.

\begin{figure}[htb]
  \centering
  \begin{tikzcd}
    \text{か} \arrow[r,"\text{符号位置に変換}"]
    &[6\zw] \text{U+304B} \arrow[r,"\text{符号単位に変換}"]
    &[6\zw] \text{\texttt{E3 81 8B}}
  \end{tikzcd}
  \caption{文字の符号単位への変換}
  \label{figure:character_to_byte_sequence}
\end{figure}

\cref{table:unicode_code_point}にUnicodeの符号化文字集合を一部抜粋して示す.

\begin{table}[htb]
  \centering
  \caption{符号化文字集合の一部}
  \label{table:unicode_code_point}
  \begin{tabular}{c|ccccccc} \hline
    U+   &      0     &      1      &      2     &      3     &      4     &      5     &      6     \\ \hline
    0020 & \UTF{0020} & \UTF{0021}  & \UTF{0022} & \UTF{0023} & \UTF{0024} & \UTF{0025} & \UTF{0026} \\
    0030 & \UTF{0030} & \UTF{0031}  & \UTF{0032} & \UTF{0033} & \UTF{0034} & \UTF{0035} & \UTF{0036} \\
    0040 & \UTF{0040} & \UTF{0041}  & \UTF{0042} & \UTF{0043} & \UTF{0044} & \UTF{0045} & \UTF{0046} \\ \hline
  \end{tabular}
\end{table}

例として「M系列」という文字列について考えよう.
「M系列」の各文字をUnicodeの符号位置に置き換えると次のようになる.
\begin{codeblock}
U+004D U+7CFB U+5217
\end{codeblock}

そして,これらをUTF-8で符号単位の列に変換すると次のようになる
\footnote{厳密には,UTF-16とUTF-32では符号単位の列をバイト列に変換する方法(\termdef{文字符号化スキーム}{もじふごうかすきーむ})も問題になる.UTF-8では,符号単位を並べたものをそのままバイト列とする.}.
「\inlinecode{4D}」がU+004D,「\inlinecode{E7 B3 BB}」がU+7CFB,「\inlinecode{E5 88 97}」がU+5217にそれぞれ対応する.
\begin{codeblock}
4D E7 B3 BB E5 88 97
\end{codeblock}

\begin{comment}
\section{日本語\TeX と文字コード}
日本語を使える\TeX 処理系は複数ある.\pTeX は

\pTeX は\termdef{JIS X 0208}{jisx0208}に含まれる文字に対応する.

\begin{enumerate}
  \item \pTeX はJIS X 0208に含まれる文字に対応する
  \item \upTeX ,\XeTeX ,\pdfTeX ,および\LuaTeX はUnicodeに含まれる文字に対応する
\end{enumerate}
\end{comment}

\section{ラスタ形式とベクタ形式}
コンピュータで画像を記録する方法は,\termdef{ラスタ形式}{らすたけいしき}と\termdef{ベクタ形式}{べくたけいしき}の2つに大別される.
ラスタ形式は,画像を単色の要素(\termdef{画素}{がそ})を集めたものとして記録する方法である.
これに対し,ベクタ形式は画像を図形の集まりとして記録する方法である.

ベクタ形式に比べ,ラスタ形式は拡大・縮小の影響を受けやすい.ラスタ形式の画像を拡大したときの様子を\cref{figure:raster_scaling}に示す.

\begin{figure}[htb]
  \begin{subfigure}{0.5\linewidth}
    \centering
    \tcbincludegraphics[myframe,width=5cm]{graph.png}
    \caption{画像全体}
  \end{subfigure}%
  \begin{subfigure}{0.5\linewidth}
    \centering
    \tcbincludegraphics[myframe,width=5cm]{raster_scaling.png}
    \caption{中央付近を拡大}
  \end{subfigure}
  \caption{ラスタ形式の画像を拡大したときの様子}
  \label{figure:raster_scaling}
\end{figure}

これに対し,ベクタ形式の画像は拡大しても\cref{figure:raster_scaling}のようにならない.
これは,ベクタ形式では拡大・縮小に応じて図形を描画しなおせるからである.
ベクタ形式の画像を拡大したときの様子を\cref{figure:vector_scaling}に示す.

\begin{figure}[htb]
  \begin{subfigure}{0.5\linewidth}
    \centering
    \tcbincludegraphics[myframe,width=5cm]{graph.png}
    \caption{画像全体}
  \end{subfigure}%
  \begin{subfigure}{0.5\linewidth}
    \centering
    \tcbincludegraphics[myframe,width=5cm]{vector_scaling.png}
    \caption{中央付近を拡大}
  \end{subfigure}
  \caption{ベクタ形式の画像を拡大したときの様子}
  \label{figure:vector_scaling}
\end{figure}

\section{補遺}
\subsection{文字化けの仕組み}
\begin{floatingfigure}{5cm}
  \centering
  \tcbincludegraphics[myframe,width=5cm]{mojibake.png}
  \caption{文字化けしたテキスト}
  \label{figure:mojibake}
\end{floatingfigure}

多くの場合,文字化けはテキストの文字コードをソフトウェアが誤判定したときに発生する.
たとえば,\cref{figure:mojibake}ではUTF-8で作成したテキストをShift\_JISで開いてしまっている.

文字化けを直すには,適切な文字コードでテキストを開きなおせばよい.
Visual Studio Code\footnote{\url{https://code.visualstudio.com}}などのテキストエディタは,ユーザが文字コードを選択しなおす機能を備えている.
\end{document}
