\documentclass[../index]{subfiles}

\begin{document}
\chapter{はじめに}
\TeX /\LaTeX は理工系で広く用いられているシステムであり,
特に数学に関わる分野では,文書作成のデファクトスタンダードと言える.
しかしながら,その全容について把握するのは容易ではない.

たとえば,\TeX /\LaTeX により次の数式を記述したいとする.
\[
  f(t) \sim \frac{a_0}{2} + \sum_{n=1}^\infty (a_n\cos nt + b_n\sin nt)
\]

あなたは「シグマ TeX」などと検索することで,概ね次のようなコードを書けばよいことが分かるだろう.
\begin{codeblock}
\[
  f(t) \sim \frac{a_0}{2} + \sum_{n=1}^\infty (a_n\cos nt + b_n\sin nt)
\]
\end{codeblock}

しかし,こうして課題のレポートをなんとか作れるようになったとしても,
「\TeX /\LaTeX とは」と問われると答えに詰まるのではないだろうか.

本稿は「\TeX /\LaTeX を理解して,自由に文書を作成するためのチュートリアル」となるべく書かれている.
特に,理工系学生がよく利用するであろう「表作成」と「グラフ作成」に重きをおき,
サンプルコードを多く載せることで,レポートを作るときに役立つようになっている.
本稿が\TeX /\LaTeX について,ブラックボックスとしてではない理解を得る一助になれば幸いである.

なお,本書の最新版はGitHub(\url{https://github.com/calamari-dev/texlatex})にある.
また,ライセンスは\href{https://creativecommons.org/licenses/by/4.0/deed.en}{CC BY 4.0}にしたがう.

\end{document}
