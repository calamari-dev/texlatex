\documentclass[../../index]{subfiles}

\begin{document}
\chapter{\TeX /\LaTeX とは}
\section{\TeX と\LaTeX}
\termdef{\TeX}\index{tex@\TeX}とは,Donald Knuthにより開発された組版システムである.
\termdef{\LaTeX}\index{latex@\LaTeX}とは,Leslie Lamportにより開発された\TeX のマクロである.

\begin{codeblock}
Hello, plain \TeX !
\bye
\end{codeblock}

この他にも,\ConTeXt というマクロパッケージがあったりする.

\LaTeXe の話も.

\section{\TeX と和文}
\termdef{\pTeX}\index{ptex@\pTeX}はJIS なんとかの文字集合に対応した\TeX 処理系である.
\termdef{\upTeX}\index{uptex@\upTeX}は内部コードをUnicodeにした\pTeX の拡張である.
\termdef{\LuaTeX}\index{luatex@\LuaTeX}は\TeX 処理系を汎用プログラミング言語である\termdef{Lua}\index{lua@Lua}によって
拡張できるようにした\TeX 処理系である.

\section{補遺}
\subsection{\SATySFi について}
\termdef{\SATySFi}\index{satysfi@\SATySFi}は,和文に対応した非\TeX 系の組版システムである.
\SATySFi を利用するという選択肢もある.

\subsection{\eTeX について}
実は,\upLaTeX を起動したときに実行されるのは,元々の\upTeX とは少し異なる\TeX 処理系である.

コマンドラインで\inlinecode{uplatex}を実行すると,次のメッセージが出力される.
\begin{codeblock}
This is e-upTeX, Version 3.141592653-p3.9.0-u1.27-210218-2.6 (utf8.uptex) (TeX Live 2021/W32TeX) (preloaded format=uplatex)
 restricted \write18 enabled.
**
\end{codeblock}

本書では以降,\eTeX 拡張の有無についてはいちいち断らないことにする.
すなわち,\upTeX と言ったら\eupTeX を指すものとする.

\subsection{他の命令の引数にできない命令}
\TeX の仕様により,\inlinecode{\verb}など一部の命令は,他の命令の引数にできない.
たとえば,次の\LaTeX ソースは\inlinecode{\verb}が
\inlinecode{\footnote}の引数になっているので不正である.
\begin{codeblock}
\documentclass[uplatex,dvipdfmx]{jsarticle}
\begin{document}
\TeX のロゴ(\emph{\verb!\TeX!で出力できる})は,Donald Knuthがこう表記するように求めている.
\end{document}
\end{codeblock}

実際,\inlinecode{example.tex}を上記の内容で作成し,
次のコマンドを実行すると「\inlinecode{LaTeX Error: \verb illegal in command argument.}」というエラーが出力される.
\begin{codeblock}
> uplatex -kanji=utf8 -no-guess-input-enc example
\end{codeblock}

この問題を解決する手っ取り早い方法は,\inlinecode{\verb}を使わないことである.
すなわち,\textbackslash を\inlinecode{\textbackslash}に置き換えればよい.
\begin{codeblock}
\documentclass[uplatex,dvipdfmx]{jsarticle}
\begin{document}
\TeX のロゴ(\emph{\texttt{\textbackslash TeX}で出力できる})は,Donald Knuthがこう表記するように求めている.
\end{document}
\end{codeblock}

\end{document}
