\documentclass[../../index]{subfiles}

\begin{document}
\chapter{\TeX /\LaTeX とは}
\section{\TeX とは}
\TeX とはDonald Knuthにより開発された組版システムである.

\section{\TeX 処理系}
\upTeX は内部コードをUnicodeにした\pTeX の拡張である.

\begin{comment}
\begin{figure}
  \centering
  \begin{tikzpicture}[graphs/every graph/.style={edges=rounded corners}]
    \matrix[row sep=4mm,column sep=7mm] {
        % Row 1:
        &
        & \node (LuaTeX) [mapping] {Lua\TeX};
        &
        & \\
        % Row 2:
        \node (TeX) [terminal] {.tex};
        & \node (upTeX) [mapping] {up\TeX};
        & \node (DVI) [terminal] {.dvi};
        & \node (dvipdfmx) [mapping] {dvipdfmx};
        & \node (PDF) [terminal] {.pdf}; \\
        % Row 3:
        &
        &
        & \node (PNG) [terminal] {.png};
        & \\
        % Row 4:
        &
        &
        & \node (dvisvgm) [mapping] {dvisvgm};
        & \node (SVG) [terminal] {.svg}; \\
     };
     \graph [use existing nodes] {
        TeX -- upTeX -> DVI -- dvipdfmx -> PDF;
        TeX --[vh path] LuaTeX ->[hv path] PDF;
        DVI --[vh path] dvisvgm -> SVG;
        PNG -- dvipdfmx;
     };
   \end{tikzpicture}  
  \caption{参考文献を\TeX で管理しない場合のフローチャート}
\end{figure}
\end{comment}

\section{\LaTeX とは}
\LaTeX とはLeslie Lamportにより開発された\TeX のマクロである.

\section{補遺}
\subsection{\eTeX について}
実は,\upLaTeX を起動したときに実行されるのは,元々の\upTeX とは少し異なる\TeX 処理系である.

コマンドラインで\inlinecode{uplatex}を実行すると,次のメッセージが出力される.
\begin{codeblock}
This is e-upTeX, Version 3.141592653-p3.9.0-u1.27-210218-2.6 (utf8.uptex) (TeX Live 2021/W32TeX) (preloaded format=uplatex)
 restricted \write18 enabled.
**
\end{codeblock}

本書では以降,\eTeX 拡張の有無についてはいちいち断らないことにする.
すなわち,\upTeX と言ったら\(\varepsilon\)-\upTeX を指すものとする.
\end{document}
